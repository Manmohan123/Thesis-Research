\documentclass[usenames,dvipsnames]{beamer}


\usepackage{soul}

\usepackage{tikz}
\usepackage{pgfplots}

\usepackage{subfigure}




\usepackage{xspace}
\usepackage{graphicx}
\usepackage{epsfig}
\usepackage{verbatim}


%\usepackage{algorithm}
%\usepackage[noend]{algpseudocode}


\usepackage[ruled,vlined]{algorithm2e}

\usepackage{float}

%\usetheme{Columbus}
\usetheme{Madrid}
 \setbeamertemplate{navigation symbols}{}
 \setbeamercovered{transparent}

\newcommand{\card}[1]{\ensuremath{|#1|}}
\newcommand{\union}{\ensuremath{\cup}}

\newcommand{\ceil}[1]{\left\lceil#1\right\rceil}
\newcommand{\floor}[1]{\lfloor#1\rfloor}

\usealerttemplate{\color{red}\bf}{}

\graphicspath{{fig/}}

\title[Uncertain Scheduling]{Replicated Data Placement for Uncertain Scheduling}

\date[APDCM 2015]{APDCM 2015\\May 25th}

\author[Erik Saule]{ Manmohan Chaubey$^1$,{\bf Erik Saule}$^1$}
 
\institute[UNCC]{ mchaubey@uncc.edu, esaule@uncc.edu\\
  $^1$ University of North Carolina at Charlotte, Computer Science
}


\begin{document}

\maketitle

\begin{frame}
  \frametitle{Outline}
  \tableofcontents[subsectionstyle=hide/hide/hide]
\end{frame}

\section{Introduction}

\AtBeginSection[]
{
  \begin{frame}<beamer>
    \frametitle{Outline}
    \tableofcontents[currentsection,subsectionstyle=hide/hide/hide]
  \end{frame}
}

\subsection{Motivation}

\begin{frame}
  \frametitle{Uncertainty in Scheduling}

  \begin{itemize}
  \item Scheduling is a common tool to manage load balance
  \item Make many assumption
    \begin{itemize}
    \item Release date are known (offline/online), or release ``rates''
    \item All machines are the same (or predictabily heterogeneous)
    \item Processing times are known
    \end{itemize}
  \item All of these assumption are typically incorrect
    \begin{itemize}
    \item Release dates are unknown, release rates are not ``constant''
    \item Machines slightly differ
    \item Processing times are either unknown or estimated
    \end{itemize}
  \end{itemize}

  \pause

  \begin{center}
    \Large How can scheduling account for this?
  \end{center}
  
\end{frame}

\begin{frame}
  \frametitle{Data Placement}
  
  \begin{itemize}
  \item If there was cost-free migration, many techniques apply:
    \begin{itemize}
    \item Having a shared work queue
    \item Workstealing
    \end{itemize}
  \item Data locality makes the difference
    \begin{itemize}
      \item If you do not have the data, you can not compute
      \item You need to access them somehow
    \end{itemize}
  \item In many applications tasks are ``pinned'' to a particular machine
    \begin{itemize}
    \item with dynamic data (i.e. Particle in Cell) little can be done
    \item opportunities with static data (Hadoop, DOoC+LAF, Out of Core)
    \end{itemize}
  \end{itemize}

  \pause
  
  \begin{block}{This paper}
    \begin{itemize}
    \item Where to place the (mostly static) data to manage uncertainty?
    \item Can replicating the data of some tasks help ?
    \end{itemize}
  \end{block}
\end{frame}

\section{Model}

\subsection{Notations}

\begin{frame}
  \frametitle{Notations}


  one more slide to justify $\alpha$?
\end{frame}


\subsection{2 Phases Algorithms}

\begin{frame}
  \frametitle{2 Phases Algorithms}


  Note: Other kinds might be better
\end{frame}


\section{Strategy 1: No Replication}

\subsection{Lower Bound}

\begin{frame}
  \frametitle{No better than $\alpha^2$}

  picture
\end{frame}

\subsection{LPT-No choice}

\begin{frame}
  \frametitle{LPT-No choice}

  \begin{block}{Algorithm}
  \end{block}

  \begin{block}{LPT-No choice is a foo competitive algorithm}
  \end{block}
\end{frame}

\section{Strategy 2: Replicate Everywhere}

\subsection{LPT-No Restriction}

\begin{frame}
  \frametitle{LPT-No Restriction}

  \begin{block}{Algorithm}
  \end{block}
  
  \begin{block}{LPT-No Restriction is a foo competitive algorithm}
  \end{block}  
\end{frame}

\section{Strategy 3: Replicate in Groups}

\subsection{LS-Groups}

\begin{frame}
  \frametitle{LS groups}

  \begin{block}{Algorithm}
    Phase 1:
    \begin{itemize}
    \item Partition the machines in $k$ groups
    \item Allocate the task to the groups with List Scheduling
    \item Replicate the data of each task on all the groups it is allocated to
    \end{itemize}
    
    Phase 2:
    \begin{itemize}
    \item Use List Scheduling in each group
    \end{itemize}    
  \end{block}
  
  \includegraphics[width=\textwidth]{figs/model3.pdf}
\end{frame}

\begin{frame}
  \frametitle{LS groups is $ \frac{k\alpha^{2}}{\alpha^{2}+k-1} (1+
  {\frac{k-1}{m}} ) + \frac{m-k}{m}$-competitive}

\end{frame}



\section{Conclusion}

\subsection{Summary}

\begin{frame}
  \frametitle{All results}
  
  \begin{center}

    \begin{tabular}{|l|c|c|c|c|c|}
      \hline
      Replication & Approximation ratio  \\
      \hline
      $|M_j|=1$ & $\frac{C_{max}}{C_{max}^{*}}\leq \frac{2\alpha^{2}m}{2\alpha^{2}+ m-1}$ \\
      & No approximation better than $\frac{\alpha^{2}m }{\alpha^{2} + m-1}$  \\
      
      \hline
      $|M_j|=m$ & $\frac{C_{max}}{C_{max}^{*}} \leq 1 + (\frac{m-1}{m})\frac{\alpha^{2}}{2}$  \\
      & $\frac{C_{max}}{C_{max}^{*}} \leq 2-\frac{1}{m}$ [Graham66]   \\
      \hline
      
      $|M_j|= \frac{m}{k} $ & $\frac{C_{max}}{C_{max}^{*}} \leq \frac{k\alpha^{2}}{\alpha^{2}+k-1} \left(1+ {\frac{k-1}{m}} \right)+ {\frac{m-k}{m}}$ \\
      
      \hline
    \end{tabular}
  \end{center}
\end{frame}

\begin{frame}
  \frametitle{All results - With numbers}

  \vspace{-1.5em}

  \begin{columns}
    \column{.4\linewidth}

    \begin{center}
      \includegraphics[width=\textwidth]{figs/alpha_11.pdf}
      
      {\footnotesize $m=210$, $\alpha=1.1$}
    \end{center}
    
    \column{.2\linewidth}

    \column{.4\linewidth}
  
    \begin{center}
      \includegraphics[width=\textwidth]{figs/alpha_15.pdf}

      {\footnotesize $m=210$, $\alpha=1.5$}
    \end{center}
    
  \end{columns}

  \vspace{-1.1em}

  \begin{columns}
    \column{.4\linewidth}
    
    \begin{center}
      \includegraphics[width=\textwidth]{figs/alpha_2.pdf}
  
      {\footnotesize $m=210$, $\alpha=2$}
    \end{center}

    \column{.6\linewidth}
    \begin{block}{Remarkables}
      \begin{itemize}
      \item Tradeoffs are found by LS-Groups
      \item LS-Groups might benefit from LPT for low $\alpha$
      \end{itemize}
    \end{block}
  \end{columns}
  
\end{frame}

\subsection{Conclusion and Future Works}

\begin{frame}
  \frametitle{Conclusion and Future Works}

  \begin{block}{Conclusion}
    
    What does it all mean?
  \end{block}

  \begin{block}{Future works}
    \begin{itemize}
      \item Can more complex strategy help?
      \item Better lower bounds? (In particular for replicate everywhere)
      \item Some replication are more ``expensive''.
    \end{itemize}
  \end{block}
\end{frame}



\begin{frame}
  \frametitle{Thank you}

    \begin{block}{More information}
    contact : esaule@uncc.edu
    
    visit: \url{http://webpages.uncc.edu/~esaule}
  \end{block}

\end{frame}


\end{document}
