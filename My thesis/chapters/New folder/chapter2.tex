\bodychapter{Strategy 2: Replicate data everywhere}\label{sec4}
\label{Intro}

With this strategy, we put no restriction on phase 2. The tasks are
replicated everywhere i.e. $\forall j, |M_{j}|=|M|$. We introduce the
\textbf{LPT-No Restriction} which replicates the data of all the tasks
on each machine in the first phase. In the second phase we simply use
the Longest Processing Time algorithm (LPT) in an online fashion using
the estimated processing time of the task. That is to say, the tasks
are sorted in non-increasing order of their estimated processing
time. Then the task are greedily allocated on the first
processor that becomes available. Note that this is done in phase 2,
the processor become available with when the actual processing time of
the task scheduled onto it elapse.

\begin{lemma}\label{No Restriction}
  Let $l$ be the task that reaches $C_{max}$ in the solution
  constructed by \textbf{LPT-No Restriction}. If there are at least two
  tasks on the machine that executes $l$ in \textbf{LPT-No Restriction}, then 
  $C_{max}^* \geq {\frac{2}{\alpha^{2}}} p_l$.
\end{lemma}
\begin{proof}
  Since there are at least two tasks on the machine that executes $l$
  in \textbf{LPT-No Restriction}, there are at least $m+1$ tasks $i$
  such that $\tilde{p_j} \geq \tilde{p_l}$. Therefore in any solution
  at least one machine gets two tasks $c$ and $d$, such that $\tilde
  p_c \geq \tilde p_l$ and $\tilde p_d \geq \tilde p_l$. $C_{max}^{*}$
  must be greater than sum of the processing time of these two tasks.
   \begin{equation}\nonumber
    C_{max}^{*}\geq p_c + p_d
  \end{equation}	

  As the actual processing time of a task must be greater than  $\frac{1}{\alpha}$ times of its estimated value, we have $p_c \geq \frac{1}{\alpha}\tilde{p_c}$ and $p_d \geq \frac{1}{\alpha}\tilde{p_d}$. Using this
  \begin{equation}\nonumber 
    C_{max}^{*} \geq \frac{1}{\alpha}\tilde p_c +  \frac{1}{\alpha} \tilde p_d \geq \frac{2}{\alpha}\tilde p_l
  \end{equation}
Since, $\tilde p_l \geq \frac{1}{\alpha} p_l$, we have
  \begin{equation}\nonumber
    C_{max}^{*} \geq {\frac{2}{\alpha^{2}}} p_l 
  \end{equation}
\end{proof}

\begin{theorem}
  \label{th:strategy2}
  \textbf{LPT-No Restriction} has a competitive ratio of
  $\frac{C_{max}}{C_{max}^{*}} \leq 1 + (\frac{m-1}{m})
  \frac{\alpha^{2}}{2}$
\end{theorem} 

\begin{proof}
  The optimal makespan, $C_{max}^{*}$ must be at least equal to the
  average load on the $m$ machines. We have
  \begin{equation}\label{eq7}
    C_{max}^{*}\geq\frac{\sum p_j}{m}
  \end{equation}

  By the property of LPT (actually, it is a property of List
  Scheduling which LPT is a refinement of) the load on each machine
  $i$ is greater than the load on the machine which reach $C_{max}$
  before the last task $l$ is scheduled. So for each machine $i$,
  $C_{max} \leq \sum_{j \in E_i}^{}{p_j} + p_l$ holds true.  Summing
  for all the machines we have
  \begin{equation}\nonumber
    mC_{max} \leq  \sum {p_j} + (m-1)p_l
  \end{equation}
  \begin{equation}\label{eq8}
    C_{max} \leq  \frac{\sum {p_j}}{m} + \frac{(m-1)}{m}p_l
  \end{equation}
  
  Using~\ref{eq7} and~\ref{eq8}, we have
  \begin{equation}\nonumber
    \frac{C_{max}}{C_{max}^{*}} \leq 1 + {\frac{m-1}{m}}\left(\frac{p_l}{C_{max}^{*}}\right)
  \end{equation}
  
  Using Lemma~\ref{No Restriction}, we have 
  \begin{equation}\nonumber
    \frac{C_{max}}{C_{max}^{*}} \leq 1 + \left(\frac{m-1}{m}\right)\frac{\alpha^{2}}{2}
  \end{equation}

\end{proof}  

Graham's List Scheduling algorithm always has a competitive ratio
of $2-\frac{1}{m}$. For $\alpha^2 < 2$, the \textbf{LPT-No
  Restriction} algorithm has better approximation than List
Scheduling. For $\alpha^2 > 2$ List Scheduling has better guarantee
than the one expressed in Theorem~\ref{th:strategy2}. Since
\textbf{LPT-No Restriction} is a variant of List Scheduling,
the algorithm has a competitive ratio of $\min (1 +
\frac{m-1}{2m}\alpha^{2}, 2-\frac{1}{m})$.