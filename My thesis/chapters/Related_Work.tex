\bodychapter{Related Work}\label{ch3}
\label{Intro}
This chapter provides the literature review on related research areas such as uncertainty in scheduling, data placement and replication. For better understanding the core concept of this thesis research proofs of some classical scheduling algorithms is presented along with a brief introduction in the context they appear while literature review.

\bodysection{Classical Scheduling problem}

 When $\alpha = 1$, the problem is exactly the classical independent
 tasks scheduling problem on identical machines, which is known to be
 NP-Hard~\cite{GareyJohnson79}. We use Graham's List Scheduling
 (LS)~\cite{Graham66} and Largest Processing Time (LPT)
 algorithms~\cite{Graham69boundson} to derive approximation ratios in
 different scenarios. The LS algorithm takes tasks one at a time and
 assigns them to the processor having the least load at that time. LS is a
 2-approximation algorithm and is widely used in online scheduling
 problems. LPT sorts the tasks in a non-increasing order of processing time and
 assigns them one at a time in this order to the processor with the
 smallest current load. The LPT algorithm has a worst case approximation
 ratio of $\frac{4}{3}-\frac{1}{3m}$ in the offline setting. One can
 even obtain an arbitrarily good approximation algorithm for this problem by increasing
 its complexity with a dual approximation
 algorithm~\cite{Hoch87}. 
 
 We begin with by recalling the formal proofs of the guarantees of LS and LPT algorithms:
 \begin{property}~\citet{Graham66} List Scheduling has an approximation ratio of $2-\frac{1}{m}$.
 \end{property}
 \begin{proof}
 Let $l$ be the last task in the system which is processed on machine $r$ and it starts on $r$ at time $t$. Clearly, the makespan $C_{max}$ of the schedule is $t+p_l$. As in LS a new task is scheduled on the least loaded machine at that time, for each machine $i$, we have $t\leq \sum\limits
 _{j \in E_i}p_j$. Adding this for all the machines including  $r$, we get $m t\leq \sum\limits_{i\in M-\{r\}}\sum\limits_{j \in E_i}p_j +\sum\limits_{j \in E_r}p_j - p_l\Rightarrow t\leq (\sum\limits_{j}p_j - p_l)/ m $. Hence, $C_{max}\leq \frac{\sum_{j}p_j + (m-1)p_l}{m}$.
  
 The optimal makespan of a schedule $C^*_{max}$ must be greater or equal to the average load over all the $m$ machines, $C^*_{max}\geq \frac{\sum_{j}p_j}{m}$. Also, $C^*_{max}$ cannot be smaller than any task in the system, hence $C^*_{max}\geq Max p_j \geq p_l$.  Therefore,  $C_{max} \leq C^*_{max}+C^*_{max}(m-1)/m $. Hence, $C_{max}/C^*_{max} \leq 2-1/m$.
  \end{proof}
  
   \begin{property}~\citet{Graham69boundson} The LPT algorithm has an approximation ratio of $\frac{4}{3}-\frac{1}{3m}$.
   \end{property}
   \begin{proof}
   LPT always generates an optimal schedule if no machine has more than 2 tasks. So to derive an approximation ratio we can assume that there are at least 3 tasks in a machine.
   As LPT assigns tasks to machines in non-increasing order of their processing times, the last task $l$ is the smallest task in the machine. Since there are at least 3 tasks in a machine $C^*_{max}\geq 3 p_l$.
   Also, $C^*_{max} \geq \frac{\sum_{j}p_j}{m}$ and $C_{max}\leq \frac{\sum_{j}p_j + (m-1)p_l}{m} $ as shown in previous proof. Therefore,  $C_{max} \leq C^*_{max}+C^*_{max}(m-1)/3m $.  Hence, $C_{max}/C^*_{max} \leq 4/3-1/3m$. 
   \end{proof}
 
 \bodysection{Uncertainty and Robustness}
 Based on various models for describing the uncertain input parameter,
 various methodologies can be used including reactive, stochastic,
 fuzzy and robust approach~\cite{DBLP:journals/cce/LiI08}. We are using
 the bounded uncertainty model which assumes that an input parameter
 have value between a lower and upper bound.  Wierman and
 Nuyens~\cite{conf/sigmetrics/WiermanN08} introduce SMART, a
 classification to understand size-based policies and draw analytic
 co-relation between response time and estimated job size in single
 server problem. Robust approaches to deal with uncertainty are widely
 used on MapReduce
 systems~\cite{Kavulya:2010:ATP:1844765.1845224}~\cite{Verma:2011:AAR:1998582.1998637},
 in
 Hadoop~\cite{Wolf:2010:FSA:2023718.2023720}~\cite{White:2009:HDG:1717298},
 on databases~\cite{Lipton199518} and on web
 servers~\cite{Cardellini99dynamicload}. The HSFS and FLEX schedulers
 provide robustness in scheduling against uncertain job
 size~\cite{Wolf:2010:FSA:2023718.2023720,6691554}. Cannon and
 Jeannot~\cite{cj09c} analyzed the correlation between various metrics
 used to measure robustness and provided scheduling heuristics that
 optimizes both makespan and robustness for scheduling task graph on
 heterogeneous system.
 
 Most of the work on robust scheduling use scenarios to structure
 the variability of uncertain parameters. Daniels and
 Kouvelis~\cite{citeulike:8334169} used them to optimize the flow-time
 using a single machine. Davenport, Gefflot, and Bek analyzed slack
 based technique (adding extra idle time) to cope with
 uncertainty~\cite{Davenport_slack-basedtechniques}. Gatto and Widmayer
 derives bounds on competitive ratio of Graham’s online algorithm in
 scenario where processing times of jobs either increase or decrease
 arbitrarily due to perturbations~\cite{Gatto07}.  These works
 considered augmenting or decreasing of job processing times as
 different problem scenario that need to be optimized. We 
 approach the problem using worst case analysis where some tasks may
 increase and some may decrease within the same schedule.
   \bodysection{Data placement and Replication}
 Data placement and replication methodologies are highly used in
 distributed systems including peer-to-peer and Grid systems to achieve
 effective data management and improve
 performance~\cite{Cirne2007213}\cite{Abawajy}\cite{4215379}. Tse~\cite{DBLP:journals/tc/Tse12}
 used selective replication of documents to increase the available
 bandwidth to serve files using web servers and study the problem
 through bi-criteria optimization techniques to maximize the quality of
 service and minimize the memory occupation. Our approach for bi- criteria optimization of makespan and memory consumption is based on the $ SBO_{\triangle}$ Algorithm proposed by Saule, Dutot and Mounie~\cite{10.1109/IPDPS.2008.4536292}. The algorithm uses  $\rho_1$ and $\rho_2$ approximated independent schedules on makespan and memory consumption respectively, and it computes a $((1+\triangle)\rho_1 , (1+\frac{1}{\triangle})\rho_2 )$- approximated schedule with $\triangle$ as a parameter of the algorithm.
 \begin{property}\citet{10.1109/IPDPS.2008.4536292}
 The $SBO_\triangle$ Algorithm generates a $ (1+\triangle)  \rho_1$-approximated schedule on makespan.
 \end{property}         
 \begin{proof}
The algorithm schedules a task $j$ according to a $\pi_2$ schedule generated by the $\rho_2$-approximated algorithm on memory if it satisfies  this condition: $\frac{{p_j}}{{C}^{\pi_1}_{max}} \leq \triangle \frac{s_j}{M^{\pi_2}_{max}}$.  Where ${C}^{\pi_1}_{max}$ is the makespan obtained using a $\pi_1$ schedule generated by the  $\rho_1$-approximated algorithm on makespan, and ${M}^{\pi_1}_{max}$ is the memory consumption of the most occupied machine obtained using $\pi_2$.  If this condition is not satisfied the task is scheduled according $\pi_1$.
 Let $k$ be the machine reaching makespan $C_{max}$ of the schedule generated by the $SBO_\triangle$ algorithm. Let  $S_1$ be the set of tasks scheduled according $\pi_1$ and $S_2$ be the set of tasks scheduled according $\pi_2$ schedule.  $C_{max}$ can be decomposed as the sum of the processing times of the tasks in set $S_1$ and $S_2$ scheduled on machine $k$.
                        \begin{equation}\nonumber
                  C_{max}= \sum_{j \in S_1 \cap E_k}^{}p_j+\sum_{j \in S_2 \cap E_k}^{}p_j 
                        \end{equation}                   
 Since $C^{\pi_1}_{max} \geq \sum\limits
                        _{j \in S_1 \cap E_k}^{}p_j$ and $\sum\limits
                        _{j \in S_2\cap E_k}\triangle {{C}^{\pi_1}_{max}} \frac{s_j}{M^{\pi_2}_{max}}\geq \sum\limits
                        _{j \in S_2\cap E_k}{p}_j $ by definition of $S_2$, we have
        \begin{equation}\nonumber
    C_{max}\leq C^{\pi_1}_{max}+\sum_{j \in S_2\cap E_k}^{}\triangle {{C}^{\pi_1}_{max}} \frac{s_j}{M^{\pi_2}_{max}}
                               \end{equation}        
      Since $\sum\limits_{j \in S2\cap E_k} \frac{s_j}{M^{\pi_2}_{max}}\leq 1$, we have
 \begin{equation}\nonumber                       C_{max}\leq(1+\triangle){C}^{\pi_1}_{max}                \end{equation}
Since ${C}^{\pi_1}_{max} \leq \rho_1 {C}^{*}_{max}$, the algorithm has an approximation ratio of $ (1+\triangle)  \rho_1$ on the makespan.
\end{proof}
\begin{property}\citet{10.1109/IPDPS.2008.4536292}
  The $SBO_\triangle$ Algorithm generates a $ (1+\frac{1}{\triangle})  \rho_2$-approximated schedule on memory.
  \end{property}         
   \begin{proof}
   Let $k $ be the machine with most memory consumption.  Similar to the previous proof, $M_{max}$ can be written as the sum of memory usage of the tasks in sets $S_1$ and $S_2$ scheduled on machine $k$, $\sum\limits
   _{j \in S_1 \cap E_k}^{}s_j+\sum\limits_{j \in s_2 \cap E_k}^{}p_j $. Since,  $\sum\limits
      _{j \in S_2 \cap E_k}^{}s_j\leq {M}^{\pi_1}_{max}$ and by definition of $ S_1$, $\sum\limits
            _{j \in S_1 \cap E_k}^{}s_j\leq \frac{{M}^{\pi_1}_{max}}{\triangle C^{\pi_2}_{max}}\sum\limits _{j \in S_2\cap E_k} p_j \leq  \frac{{M}^{\pi_1}_{max}}{\triangle }$, we have
           \begin{equation}\nonumber                     M_{max}\leq(1+\frac{1}{\triangle}){M}^{\pi_1}_{max}                         \end{equation}
       Since ${M}^{\pi_1}_{max} \leq \rho_2 {M}^{*}_{max}$, the algorithm has an approximation ratio of $ (1+\frac{1}{\triangle})  \rho_2$ on memory.
                               \end{proof}
                           
                           
  
 